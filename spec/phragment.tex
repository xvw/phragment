\documentclass[a4paper,10pt]{article}
\usepackage[french]{babel} 
\usepackage[utf8x]{inputenc}
\usepackage[T1]{fontenc}
\usepackage{lmodern}
\usepackage{makeidx}
\usepackage{multicol}
\usepackage{hyperref}
\usepackage{pdfpages}

\newcommand{\etal}{\textit{et al.}}
\newcommand{\ahref}[3][blue]{\href{#2}{\color{#1}{#3}}}

\title{
  \includegraphics[width=2cm]{../iconography/phragment-black.png}\\
  \bsc{Phragment}\\
  \large{
    Une tentative protocole pour des conversations décentralisées
    où tout le monde est propriétaire de son contenu\\~\\~\\
    \textbf{version} \texttt{0.0.1}
  }
}

\date{}
\author{
  Xavier \bsc{Van de Woestyne}\\
  \texttt{xaviervdw@gmail.com}\\
  \small{https://xvw.github.io}
}


\hypersetup{ hidelinks, }
\addto{\captionsfrench}{\renewcommand{\abstractname}{}}

\begin{document}

\maketitle

\begin{abstract}
  A l'heure ou le web décentralisé revient à la tendance, pour de multiples
  raisons valables (écologiques, morales et idéologiques, ouvertures de
  perspectives), on trouve beaucoup de solutions qui pallient à des soucis
  liés à l'excès de centralisation dans le web. \bsc{Phragment} est un
  protocole dont l'objectif initial est de servir un système de commentaires
  pour une application web statique, tâchant de permettre aux intervenants
  de contrôler leurs différents contenus. Le protocole ne décrit aucune
  innovations particulières et souffre d'un manque d'ergonomie flagrant,
  cependant, j'assume parfaitement le plaisir de réinventer, encore une fois,
  une roue, bancale et peu robust. L'objectif de ce document est de survoler
  les motivations (et le contexte) d'un tel protocol, son formalisme,
  l'élaboration d'un client de référence, le survol de certains cas d'usages,
  les améliorations possibles, points de blocages et faiblesses intrinsèques.
  \\~\\
  Ce document est assez informel, il repose avant tout sur un besoin
  personnel, et est rédigé dans un style bien peu générique. Pour tout lecteurs
  potentiels, prenez donc ce document comme une feuille de route, rédigée
  au long de l'implémentation du protocole et de ses cas d'usages.
  \\
\end{abstract}

Pour faciliter l'hébergement de ma page personnelle, j'ai fais le choix d'utiliser un
système de génération statique. Même si la génération de site statique couvre une
grande partie de mes besoins, je trouvais ça amusant d'avoir un espace de
commentaires en dessous de chacun de mes articles. Pour enrichir une collection de
pages statiques de fonctionnalités dynamiques, une approche commune est de passer
par un tier. Soit via un greffon, comme \ahref{https://disqus.com/}{Disqus}, soit
via l'invitation explicite à changer d'emplacement, proposant un espace de discussion
sur un fil \ahref{https://www.reddit.com/}{Reddit},
\ahref{https://news.ycombinator.com/}{Hacker news} ou autre service apparent.\\
Pour ma page personnelle, j'ai utilisé Disqus. Bien que le service soit relativement
bien pensé (et offre une UI suffisamment générique pour s'intégrer dans beaucoup
d'interfaces graphiques), le service souffre tout de même (de manière assez
subjective) de plusieurs soucis:\\

\begin{itemize}
\item le service centralise tous les commentaires ;
\item le mainteneur du site n'a pas de contrôle sur le cercle communautaire mis en place ;
\item aucune personnalisation réelle envisageable ;
\item l'outils est propriétaire et on ne dispose pas d'outils pour comprendre à quelles
  fins sont utilisées les données.\\
\end{itemize}

L'ensemble de ces soucis m'ont amené à réfléchir à une autre solution pour le développement
de Planet\cite{planet}, la platforme (en devenir) que je développe pour la refonte
de mon espace personnel sur internet.

\newpage
\bibliographystyle{unsrt}
\bibliography{phragment.bib}

\end{document}

